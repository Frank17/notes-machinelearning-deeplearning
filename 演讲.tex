\documentclass[a4paper, 12pt]{article}
\usepackage{xeCJK}
\usepackage{amsmath}
\usepackage{hyperref}
\usepackage{amsfonts}
\usepackage{amssymb}
\usepackage{graphicx}
\setlength{\parindent}{24pt}
\title{我和我的祖国}
\author{涂梦飞}
\date{2019年10月11日}
\begin{document}
  \maketitle
  亲爱的同学们:
  
  大家好,很荣幸能够参加此次演讲,今天我的主题是:我和我的祖国。
  
  21世纪是一个腾飞的世纪,在这个世纪里,中国,一个伟大的名字,从此镌刻在了历史前进的车碾中。二〇〇一,世界为北京而喝彩、沸腾,中国二字即将在“五环”内绽放异彩;二〇〇七,宇宙为东方而歌唱,“嫦娥”怀揣着她这五千多年的梦想跨入了那个闪亮的神话;二〇〇八,灾难撼得动古老的大地,但摧毁不了民族的骨血,生命奇迹在废墟中发芽,迸发的是生生不息的民族之魂;二〇一〇,寒冬裹不住火热的双手,一双名为“中国制造”的手打造的是民族的脊梁,是民族复兴的基石;二〇一二,一八九四的海洋将由“辽宁号”激起千层的巨浪,踏出最“致远”的征途。
  
  我,在一个衔接了祖国荣辱的年代,诵读了这首浩浩汤汤的富强史诗。

  七十年前,饱经战火洗礼的中国人民站立起来,14年抗战,浴血奋战的中国人民摆脱了受尽奴役的屈辱岁月,可铁蹄之下我看见的是中国人民的铮铮铁骨,绵延万里的长城虽固若金汤,但前赴后继的无畏铸就了永恒不衰的铁骨千万,其比金坚,想必始皇帝也为之折服。鸭绿江外,雄赳赳气昂昂,小米步枪保卫着新生的共和国血脉。这是我的祖国,一幅不屈淫威的钢铁之躯,一颗至死不渝的赤子之心。

  荒漠中,遍地都是死亡的恐惧,任何一切都显得苍白。怎可奈“一阵惊雷卷飓风,万里红”%\footnote{来自毛泽东《满江红·庆祝第一次核试验》}
  。“596”炸毁的是紧扼咽喉的枷锁,炸出了自由。“东方红,太阳升”%\footnote{来自东方红歌词}
  ,歌声响彻太空,奏出新中国的伟岸,描绘了壮丽山河。“紫荆花”,“莲花”在长江黄河中结出了民族气节,“七子”%\footnote{取自闻一多《七子之歌》}
  再也不用哭泣,襁褓易碎,但山河仍在。

  这是20世纪的中国,我们从风雨中走出,从炮火中走出,从压迫中走出,从屈辱中走出,但中国人民走出了昂扬不屈的斗志,中华民族走出了舍我其谁的大义凛然。

  21世纪的年轮,碾出了14亿人的“中国梦”,复兴路。我目睹了鸟巢之夜的磅礴气势,见证了奥运圣火在北京跃然生起的光辉;我梦见了“嫦娥”奔月的奇妙之旅,想必,这是对莫高窟最崇高%\footnote{敦煌莫高窟飞天壁画}
  的礼赞;“尔来四万八千岁,不与秦塞通人烟”%\footnote{取自李白《蜀道难》}
  ,可5·12送来的是跨越“峥嵘剑阁”的八方支援,我惊叹的是生命的顽强,倾倒的是那一簇紧密团结的民族之花。我们跻身世界第二,我们正在向民族的伟大复兴大步迈去,我们正在“中国梦”的征程上,砥砺前行。海洋上的披荆斩棘,广袤天空的呼啸而过,人民军队带着钢铁意志,为我们构筑着一道不可逾越的天堑,一八九四的耻辱是我们脑海中振聋发聩的冲锋号角。

  祖国,和我,我们在前行,在远处,永远没有终点。

\end{document}